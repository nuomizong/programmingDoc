% !Mode:: "TeX:UTF-8"	% read in as utf8 file.

\section{File format}

\subsection{Force File}
\begin{lstlisting}[language=python]
# use Maya ExportSelectedVerts mel to export selected vetex indexes in Maya
# file extension: <name>.force
# <vi> is a integer meaning ith node of mesh.
# <fix> is x component of force at ith node.
# total: <number of forces>
<no.> <v1> <f1x> <f1y> <f1z>
...
<no.> <vn> <fnx> <fny> <fnz>
\end{lstlisting}
\subsection{Boundary File}
\begin{lstlisting}[language=python]
# use Maya ExportSelectedVerts mel to export selected vetex indexes in Maya
# file extension: <name>.bou
# <vi> is a integer meaning ith node of mesh.
# total: <number of bous>
<no.> <v1>
...
<no.> <vn>
\end{lstlisting}
\subsection{Ribs Definition}
There are two kinds of rib input: \textbf{Edge aligned ribs} or \textbf{General ribs}.
\subsubsection{Edge aligned ribs}
\begin{lstlisting}[language=python]
# use Maya ExportSelectedEdges mel to export selected edges in Maya
# file extension: <name>.edge
# <eli> is a integer meaning ith node of mesh.
# total: <number of edges>
<no.> <e11> <e12>
...
<no.> <en1> <en2>
\end{lstlisting}
\subsubsection{General ribs}
\begin{lstlisting}[language=python]
# file extension: <name>.rib
# <faceId> is an integer meaning the face a rib belongs to.
# <pt1> <pt2> are starting and ending points of the rib in the face.
# total: <number of ribs>
<no.> <faceId> <pt1x> <pt1y> <pt1z> <pt2x> <pt2y> <pt2z>
...
<no.> <faceId> <pt1x> <pt1y> <pt1z> <pt2x> <pt2y> <pt2z>
\end{lstlisting}	


\section{Cross section}
\begin{figure}[h]
	\centering
	\includegraphics[width=0.2\linewidth, angle=-90]{crossSection_Rect}
	\caption{ Rectangle Cross section}
	\label{fig:crosssectionrect}
\end{figure}

\section{Explanation for different mises}
as the material are pushing far away from rib's center, the normal way of computing as:
\begin{equation}
\sigma = D B \delta^e
\end{equation}

is not acceptable any more. The far end of material must have much larger stress than near end!