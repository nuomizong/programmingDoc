\documentclass[10pt,a4paper]{article}
\usepackage[latin1]{inputenc}
\usepackage{amsmath}
\usepackage{amsfonts}
\usepackage{amssymb}
\usepackage{graphicx}

\begin{document}
	
	
	\author{nuomizong}
	\title{Find Intersected Point of Involute curve and a Specific Circle}
	\date{19/06/2015}
	\maketitle
	
	\newpage
	
	\section{Explanation}
	Figure \ref{fig: involute curve definition} shows the principle of \textbf{Involute Curve}. In figure \ref{fig: involute curve}, \textbf{A} is the starting point. When $ |\mathrm{OA}| $ turns an anti-clockwise angle $ \theta $, the arc length of $ |\mathrm{AC}| $ should equal to the length of $ |\mathrm{CP}| $ according to the definition of \textbf{Involute Curve}. If Point P is also located on circle of radius r, then the turned angle $ \theta $ can be solved analytically.
	
	\begin{figure}[h]
		\centering
		\includegraphics[width = 0.5\textwidth]{involute_curve_definition}
		\caption{Involute curve definition} \label{fig: involute curve definition}
		\vspace{\baselineskip}
	\end{figure}
	
	\begin{figure}[h]
		\centering
		\includegraphics[width = 0.5\textwidth]{involute_curve}
		\caption{Involute } \label{fig: involute curve}
		\vspace{\baselineskip}
	\end{figure}
	
	\section{Algorithms}
	After $ |OA| $ turns an anti-clockwise angle $ \theta $, so we have:
	
	\begin{equation}
	|\mathrm{CA}| = s = r_b \theta
	\end{equation}
	
	where $ r_b $ means Base Circle.
	
	Because Point C lies on Base Circle, there we can compute its coordinates:
	
	\begin{equation}
		\begin{cases}
		x_c = r_b \cos(\theta) \\
		y_c = r_b \sin(\theta)
		\end{cases}
	\end{equation}
	
	Because $ |CA| = |CP| $, so the Point P can be calculated using:
	
	\begin{equation}
		\begin{cases}
			x_p = x_c + s \sin(\theta) \\
			y_p = y_c - s \cos(\theta)
		\end{cases}
	\end{equation}
	
	As a result of Point P lies on another circle, so we have:
	
	\begin{equation}
		x_p^2+y_p^2 = r^2
	\end{equation}
	
	Solve all aforementioned equations, we conclude:
	
	\begin{equation}
		\theta = \sqrt{(\dfrac{r}{r_b})^2-1}
	\end{equation}
	
	Finally, Point P is determined via:
	
	\begin{equation}
		\begin{cases}
			x_p = r_b \cos(\theta)+r_b\theta \sin(\theta)\\
			y_p = r_b \sin(\theta)-r_b\theta \cos(\theta)
		\end{cases}
	\end{equation}
\end{document}