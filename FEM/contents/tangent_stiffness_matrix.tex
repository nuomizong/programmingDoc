% !Mode:: "TeX:UTF-8"	% read in as utf8 file.

\chapter{Tangent Stiffness Matrix}
\section{Linear Tetrahedron}
\subsection{Basic equations}
\begin{CJK*}{UTF8}{song} % CJK*环境
线性四面体(立体)元(linear tetrahedral (solid) element)是既有局部坐标又有总体坐标的三维有限元,用线性形函数描述。它也被称为常应变四面体元。线性四面体(立体)元的系数有弹性模量$ E $和泊松比$ v $。每个线性四面体(立体)元有\num{4}个节点并且每个节点有\num{3}个自由度。这\num{4}个节点的总体坐标用$ (x_1, y_1,z_1), (x_2,y_2,z_2) $ $, (x_3,y_3,z_3) $和$ (x_4,y_4,z_4) $表示。节点的编号很重要——应该对节点进行编号以便使单元的体积为正值。建议用MATLAB的TetrahedronElementVolume函数验证。该函数专门为这一目的而编写。因此,单元刚度矩阵给定如下(参见文献[1]和[8]):

\begin{equation} \label{eq: k definition}
\mathbf{k} = V \mathbf{B}^T \mathbf{D} \mathbf{B}
\end{equation}

式中,$ V $是单元的体积,由下式给出:
\begin{equation}
6V = \begin{vmatrix}
1 & x_1 & y_1 & z_1 \\ 
1 & x_2 & y_2 & z_2 \\ 
1 & x_3 & y_3 & z_3 \\ 
1 & x_4 & y_4 & z_4
\end{vmatrix} 
\end{equation}

矩阵$ \mathbf{B} $由下式确定:
\setcounter{MaxMatrixCols}{20}
\begin{equation} \label{eq: matrix B definition}
\mathbf{B} = \begin{bmatrix}
\dfrac{\partial N_1}{\partial x} & 0 & 0 & \dfrac{\partial N_2}{\partial x} & 0 & 0 & \dfrac{\partial N_3}{\partial x} & 0 & 0 & \dfrac{\partial N_4}{\partial x} & 0 & 0 \\ 
0 & \dfrac{\partial N_1}{\partial y} & 0 & 0 & \dfrac{\partial N_2}{\partial y} & 0 & 0 & \dfrac{\partial N_3}{\partial y} & 0 & 0 & \dfrac{\partial N_4}{\partial y} & 0 \\ 
0 & 0 & \dfrac{\partial N_1}{\partial z} & 0 & 0 & \dfrac{\partial N_2}{\partial z} & 0 & 0 & \dfrac{\partial N_3}{\partial z} & 0 & 0 & \dfrac{\partial N_4}{\partial z} \\ 
\dfrac{\partial N_1}{\partial y} & \dfrac{\partial N_1}{\partial x} & 0 & \dfrac{\partial N_2}{\partial y} & \dfrac{\partial N_2}{\partial x} & 0 & \dfrac{\partial N_3}{\partial y} & \dfrac{\partial N_3}{\partial x} & 0 & \dfrac{\partial N_4}{\partial y} & \dfrac{\partial N_4}{\partial x} & 0 \\ 
0 & \dfrac{\partial N_1}{\partial z} & \dfrac{\partial N_1}{\partial y} & 0 & \dfrac{\partial N_2}{\partial z} & \dfrac{\partial N_2}{\partial y} & 0 & \dfrac{\partial N_3}{\partial z} & \dfrac{\partial N_3}{\partial y} & 0 & \dfrac{\partial N_4}{\partial z} & \dfrac{\partial N_4}{\partial y} \\ 
\dfrac{\partial N_1}{\partial z} & 0 & \dfrac{\partial N_1}{\partial x} & \dfrac{\partial N_2}{\partial z} & 0 & \dfrac{\partial N_2}{\partial x} & \dfrac{\partial N_3}{\partial z} & 0 & \dfrac{\partial N_3}{\partial x} & \dfrac{\partial N_4}{\partial z} & 0 & \dfrac{\partial N_4}{\partial x}
\end{bmatrix} 
\end{equation}

在\cref{eq: matrix B definition}中,形函数$ N_1, N_2, N_3 $和$ N_4 $由下式给出:
\begin{align}
N_1 &= \dfrac{1}{6V} (\alpha_1 + \beta_1 x + \gamma_1 y + \delta_1 z)\\
N_2 &= \dfrac{1}{6V} (\alpha_2 + \beta_2 x + \gamma_2 y + \delta_2 z)\\
N_3 &= \dfrac{1}{6V} (\alpha_3 + \beta_3 x + \gamma_3 y + \delta_3 z)\\
N_4 &= \dfrac{1}{6V} (\alpha_4 + \beta_4 x + \gamma_4 y + \delta_4 z)
\end{align}

其中参数$ \alpha_1, \alpha_2, \alpha_3, \alpha_4, \beta_1, \beta_2, \beta_3, \beta_4, \gamma_1, \gamma_2, \gamma_3, \gamma_4, \delta_1, \delta_2, \delta_3, \delta_4 $给定如下:

\begin{equation}
\begin{array}{cccc}
\alpha_1 = \begin{vmatrix}
x_2 & y_2 & z_2 \\ 
x_3 & y_3 & z_3 \\ 
x_4 & y_4 & z_4
\end{vmatrix} &  \alpha_2 = -\begin{vmatrix}
x_1 & y_1 & z_1 \\ 
x_3 & y_3 & z_3 \\ 
x_4 & y_4 & z_4
\end{vmatrix} & \alpha_3 = \begin{vmatrix}
x_1 & y_1 & z_1 \\ 
x_2 & y_2 & z_2 \\ 
x_4 & y_4 & z_4
\end{vmatrix} & \alpha_4 = -\begin{vmatrix}
x_1 & y_1 & z_1 \\ 
x_2 & y_2 & z_2 \\ 
x_3 & y_3 & z_3
\end{vmatrix}
\end{array} 
\end{equation}

\begin{equation}
\begin{array}{cccc}
\beta_1 = -\begin{vmatrix}
1 & y_2 & z_2 \\ 
1 & y_3 & z_3 \\ 
1 & y_4 & z_4
\end{vmatrix} &  \beta_2 = \begin{vmatrix}
1 & y_1 & z_1 \\ 
1 & y_3 & z_3 \\ 
1 & y_4 & z_4
\end{vmatrix} & \beta_3 = -\begin{vmatrix}
1 & y_1 & z_1 \\ 
1 & y_2 & z_2 \\ 
1 & y_4 & z_4
\end{vmatrix} & \beta_4 = \begin{vmatrix}
1 & y_1 & z_1 \\ 
1 & y_2 & z_2 \\ 
1 & y_3 & z_3
\end{vmatrix}
\end{array} 
\end{equation}

\begin{equation}
\begin{array}{cccc}
\gamma_1 = \begin{vmatrix}
1 & x_2 & z_2 \\ 
1 & x_3 & z_3 \\ 
1 & x_4 & z_4
\end{vmatrix} &  \gamma_2 = -\begin{vmatrix}
1 & x_1 & z_1 \\ 
1 & x_3 & z_3 \\ 
1 & x_4 & z_4
\end{vmatrix} & \gamma_3 = \begin{vmatrix}
1 & x_1 & z_1 \\ 
1 & x_2 & z_2 \\ 
1 & x_4 & z_4
\end{vmatrix} & \gamma_4 = -\begin{vmatrix}
1 & x_1 & z_1 \\ 
1 & x_2 & z_2 \\ 
1 & x_3 & z_3
\end{vmatrix}
\end{array} 
\end{equation}

\begin{equation}
\begin{array}{cccc}
\delta_1 = -\begin{vmatrix}
1 & x_2 & y_2 \\ 
1 & x_3 & y_3 \\ 
1 & x_4 & y_4
\end{vmatrix} &  \delta_2 = \begin{vmatrix}
1 & x_1 & y_1 \\ 
1 & x_3 & y_3 \\ 
1 & x_4 & y_4
\end{vmatrix} & \delta_3 = -\begin{vmatrix}
1 & x_1 & y_1 \\ 
1 & x_2 & y_2 \\ 
1 & x_4 & y_4
\end{vmatrix} & \delta_4 = \begin{vmatrix}
1 & x_1 & y_1 \\ 
1 & x_2 & y_2 \\ 
1 & x_3 & y_3
\end{vmatrix}
\end{array} 
\end{equation}

在方程\cref{eq: k definition}中,矩阵$ \mathbf{D} $由下式确定:
\begin{equation}
\mathbf{D} = \dfrac{E}{(1+v)(1-2v)} \begin{bmatrix}
1-v & v & v & 0 & 0 & 0 \\ 
v & 1-v & v & 0 & 0 & 0 \\ 
v & v & 1-v & 0 & 0 & 0 \\ 
0 & 0 & 0 & \dfrac{1-2v}{2} & 0 & 0 \\ 
0 & 0 & 0 & 0 & \dfrac{1-2v}{2} & 0 \\ 
0 & 0 & 0 & 0 & 0 & \dfrac{1-2v}{2}
\end{bmatrix} 
\end{equation}

很显然,线性四面体(立体)元有12个自由度——每个节点有\num{3}个自由度。因此,对一个有$ n $个节点的结构而言,其整体刚度矩阵$ K $应该是$ 3n \times 3n $的(因为每个节点有\num{3}个自由度)。调用MATLAB的TetrahedronAssemble函数可以得到整体刚度矩阵$ K $,该函数专门为这一目的而编写。后面的示例将会详细介绍该函数的用法。

一旦得到整体刚度矩阵$ K $,就可以列出以下方程组:
\begin{equation}
\mathbf{K} \vec{U} = \vec{F}
\end{equation}

式中,$ \vec{U} $是结构节点位移矢量,$ \vec{F} $是结构节点载荷矢量。在这一步中,边界条件被手动赋值给矢量$ \vec{U} $和$ \vec{F} $。然后用分解法和高斯消去法求解。最后,一旦求出未知的位移和支反力,就可以用下式求出每个单元的应力矢量。

\begin{equation}
\vec{\sigma} = \mathbf{D}\mathbf{B} \vec{u}
\end{equation}

式中,$ \sigma $是$ 6\times 1 $的单元应力矢量,$ \vec{u} $是$ 12 \times 1 $的单元节点位移矢量。每个单元的$ \sigma $形式为$ \vec{\sigma} = [\sigma_x, \sigma_y, \sigma_z, \tau_xy, \tau_yz, \tau_zx]^T $。
\end{CJK*}




