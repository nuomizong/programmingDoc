% !Mode:: "TeX:UTF-8"	% read in as utf8 file.
\documentclass[10pt,a4paper]{article}

\input{D:/Dropbox/myStyle.tex}
%\input{/Users/nuomizong/Dropbox/myStyle.tex}

\begin{document}

\author{Anzong Zheng}
\title{Programming Algorithm}
\date{February 14th, 2016}
\maketitle
\newpage

\section{Green's strain tensor}
The elastic potential energy of a body captures the amount of work required to deform the body from the rest state into the current configuration. It is expressed in terms of the \textit{strain tensor} and \textit{stress tensor}. Strain is the degree of metric distortion of the body. A standard measure of strain is Green's strain tensor [Capell 2002]:
\begin{equation}
e_{ij} = \dfrac{\partial d^i}{\partial x^j} + \dfrac{\partial d^j}{\partial x^i} + \delta_{kl} \dfrac{\partial d^k}{\partial x^i} \dfrac{\partial d^l}{\partial x^j}
\end{equation}

Another form of \textit{Green strain tensor} $ \mathbf{E} \in \mathbf{R}^{3 \times 3} $, defined as [Sifakis 2012]:
\begin{equation}
\mathbf{E} = \dfrac{1}{2} \left( \mathbf{F}^T \mathbf{F} - \mathbf{I} \right) 
\end{equation}

\section{Internal strain energy}
For hyperelastic materials, the internal strain energy [Barbic 2007] equals:
\begin{equation} \label{eq: internal strain energy}
\int_{\Omega} \mathbf{S} : \mathbf{E} dV
\end{equation}

where integration is performed over the reference (i.e. underformed) configuration $ \Omega $. Here, $ \mathbf{S} $ is the second Piola-Kirchhoff stress tensor. $ \mathbf{E} $ is the Green-Lagrange strain tensor, and colon denotes tensor contraction, i.e., element-wise dot product $ \mathbf{S} : \mathbf{E} = \sum_{i=1}^{3} \sum_{j=1}^{3} S_{ij} E_{ij} $. Equation \ref{eq: internal strain energy} follows from the fact that second Piola-Kirchhoff stress tensor is work-conjugate to $ \dot{E} $, and the path-independence of internal strain energy for hyperelastic materials.

\subsection{Linear strain energy density}
[Sifakis 2012] Linear approximation of Green strain tensor $ \mathbf{E}(\mathbf{F}) $ is denoted by \textit{small strain tensor} or \textit{infinitesimal strain tensor} $ \mathbf{\epsilon} $, where:
\begin{equation}
\mathbf{\epsilon}=\dfrac{1}{2} (\mathbf{F} + \mathbf{F}^T) - \mathbf{I}
\end{equation}
The simplest practical constitutive model is \textit{linear elasticity}, defined in terms of the strain energy density as:
\begin{equation}
\Psi(\mathbf{F}) = \mu \mathbf{\epsilon} : \mathbf{\epsilon} + \dfrac{\lambda}{2} \mathrm{tr}^2(\mathbf{\epsilon})
\end{equation}

where $ \mathbf{\epsilon} $ is the small strain tensor and $ \mu $, $ \lambda $ are the \textit{Lam\'{e} coefficients}.

The first Piola-Kirchhoff stress tensor is:
\begin{equation}
\mathbf{P}(\mathbf{F}) = \mu (\mathbf{F}+\mathbf{F}^T-2\mathbf{I}) + \lambda \mathrm{tr}(\mathbf{F}-\mathbf{I})\mathbf{I}
\end{equation}

\subsection{St. Venant-Kirchhoff model}
[Sifakis 2012] The \textit{Green strain tensor} used in StVK material is defined as:
\begin{equation}
\mathbf{E} = \dfrac{1}{2} \left( \mathbf{F}^T \mathbf{F} - \mathbf{I} \right)
\end{equation}

The constitutive model of \textit{St. Venant-Kirchhoff material} is the first truly nonlinear material. The internal strain energy density for stvk is:
\begin{equation}
\Psi (\mathbf{F}) = \mu \mathbf{E}:\mathbf{E} + \dfrac{\lambda}{2} \mathrm{tr}^2(\mathbf{E})
\end{equation}

and therefore the first Piola-Kirchhoff stress tensor:
\begin{equation}
\mathbf{P}(\mathbf{F}) = \mathbf{F}[2\mu \mathbf{E} + \lambda \mathrm{tr}(\mathbf{E})\mathbf{I} ]
\end{equation}
\section{Lam\'{e} coefficients}
Lam\'{e} coefficients are related to the material properties of \textit{Young's modulus E } (a measure of stretch resistance and \textit{Poisson's ratio }) $ v $ (a measure of incompressibility) as:
\begin{align} \label{eq:lame coefficients}
\lambda &= \dfrac{\upsilon E}{(1+\upsilon)(1-2\upsilon)} \\
\mu &= \dfrac{E}{2(1+\upsilon)}
\end{align}

\section{The equations of motion}
The \textit{unreduced} equations of motion for structural vibrations of a volumetric 3D deformable object, under the FEM discretization, can be derived from the principle of virtual work of Lagrangian mechanics. These equations of motion [Barbic 2007] (called the \textit{Euler-Lagrange equation}) are a second-order system of ordinary differential equations:
\begin{equation} \label{eq: Euler-Lagrange equation}
\mathbf{M} \ddot{\vec{u}} + \mathbf{D} \left( \vec{u}, \dot{\vec{u}}\right) + \mathbf{R}\left( \vec{u} \right) = \vec{f}
\end{equation}

Here, $ \vec{u} \in \mathbb{R}^{3n} $ is the displacement vector (the unknown), $ \mathbf{M} \in \mathbb{R}^{3n,3n} $ is the \textit{mass matrix}, $ \mathbf{D}\left(u, \dot{u}\right) \in \mathbb{R}^{3n}$ are damping forces, $ \vec{R}(\vec{u}) \in \mathbb{R}^{3n} $ are internal deformation forces, and $ \vec{f} \in \mathbb{R}^{3n} $ are any externally applied forces.

\section{Shape function}
$ \Phi_a = \Phi_a (\vec{X}) $ denotes the FEM shape function [Barbic 2007] corresponding to vertex $ a $, i.e. $ \Phi_a(a) = 1 $ and $ \Phi_a(b) = 0 $ for $ a \neq b $.

\section{General linear materials}
[Barbic 2007] In this case, the second Piola-Kirchhoff stress tensor $ \mathbf{S} $ is modelled as a linear function of Green-Lagrange strain $ \mathbf{E} $:
\begin{equation}
S_{ij}  = C_{ijkl} E_{kl}
\end{equation}

where $ S_{ij} $ is the component of $ \mathbf{S} $ at location $ (i,j) $, for $ i,j = 1,2,3 $. $ E_{kl} $ is the component of E at location $ (k,l) $, for $ k,l=1,2,3 $, and $ C_{ijkl} = C_{klij} = C_{jikl} = C_ijkl $ is the rank-4 \textit{elasticity tensor}. There are \num{36} free degrees of freedom in C. We derived the unreduced internal force on $ c $, under the deformation vector $ u \in \mathbb{R}^{3n} $, as per arising from material within element $ e $, to be

\begin{align}
f_c(u) = &\left( \dfrac{1}{2} \left( C_{ijkl} + C_{ijlk}\right) P_{lj}^{ac} e_i \otimes e_k\right) u_a + \dfrac{1}{2} \left( C_{ijkl} Q_{klj}^{abc} e_i\right) (u_a \cdot u_b) + \\
 &(u_d \otimes u_a)\left( \dfrac{1}{2}(C_{mjkl}+C_{mjlk}) Q_{lmj}^{adc} e_k\right) + \dfrac{1}{2} C_{mjkl} R_{kljm}^{abcd}(u_a \cdot u_b)u_d \\
 P_{ij}^{ab} = &\int_{e}\nabla \phi_{ai} \nabla \phi_{bj} dV \in \mathbb{R} \\
 Q_{ijk}^{abc} = &\int_{e} \nabla \phi_{ai} \nabla \phi_{bj} \nabla \phi_{ck} dV \in \mathbb{R} \\
 R_{ijkl}^{abcd} = & \int_{e} \nabla \phi_{ai} \nabla\phi_{bj} \nabla\phi_{ck} \nabla\phi_{dl}dV \in \mathbb{R}
\end{align}

Here, $ \nabla\phi_{ai} $ denotes the $ i $-th component of $ \nabla\phi_a = \nabla \phi_a(X) $, for $ i=1,2,3 $. Note that StVK is obtained for $ C_{ijkl} = \lambda \delta_{ij} \delta_{kl} + 2\mu \delta_{ik} \delta_{jl} $

\section{derived formulas} 
[Barbic 2007]First, we expressed $ \mathbf{E} = \mathbf{E}(\vec{X}) $ (at any location $ \vec{X} $ inside the material) as a quadratic function of mesh vertex deformations. Then, we applied the material law to express $ \mathbf{S} = \mathbf{S}(\vec{X}) $ as a (still quadratic due to material law linearity) function of mesh vertex deformations. Next, we used the formula $ \mathbf{P} = \mathbf{F} \cdot \mathbf{S} $ to express the first Piola-Kirchhoff stress tensor as a cubic polynomial in mesh vertex deformation. Finally, we used the Rayleigh-Ritz virtual work formulation from Lagrange mechanics which states that the generalized force corresponding to the shape function $ \phi = \phi(\vec{X}) $ equals
\begin{equation}
\vec{f} = - \int_{\Omega} \mathbf{P}:\nabla_\phi (\vec{X}) dV + \int_{\Omega}\vec{f}_{ext}(\vec{X}) \cdot \phi(\vec{X}) dV + \int_{\partial \Omega} \mathbf{P}^T\phi(\vec{X})ndS
\end{equation}

where $ \Omega $ denotes the 3D object in the rest configuration. If this principle is applied to the FEM shape function $ \phi_c $ of mesh vertex $ c $.

\section{Static simulation}
[Barbic 2007] It is easy to convert the Euler-Lagrange equation \ref{eq: Euler-Lagrange equation} into a form that supports static simulations : simply discard all terms except $ \vec{R}(\vec{u}) $ and $ \vec{f} $. Given an external static load $ \vec{f} $, the problem then becomes solving the nonlinear equation $ \vec{R}(\vec{u}) = \vec{f} $ for $ \vec{u} $. This is commonly done by the application of a Newton-Raphson procedure, which is a generalization of the familiar 1D method to solve for the roots of a 1D nonlinear equation. Newton-Raphson involves repeated iterations consisting of evaluating the internal forces $ \vec{R}=\vec{R}(\vec{u}) $ and the tangent stiffness matrix $ \mathbf{K} = \mathbf{K}(\vec{u}) $, followed by a linear system solve $ \mathbf{K} \Delta \vec{u} = \vec{f} -\vec{R} $ to obtain the next solution approximation $ \vec{u}+ \Delta \vec{u} $.

\section{Understanding of stiffness matrix}
The matrix relation between strains and nodal displacements for linear Fem is:
\begin{equation}
\vec \epsilon = \mathbf{B} \vec{u}
\end{equation}

If the material is linearly elastic and the constitutive equation may be compactly expressed as:
\begin{equation}
\vec{\sigma} = \mathbf{C} \vec{\epsilon}
\end{equation}

If the material is isotropic, with elastic modulus $ E $ and Poisson's ratio $ \nu $, the forgoing relation simplifies to:
\begin{equation}
\begin{bmatrix}
\sigma_{xx} \\
\sigma_{yy} \\
\sigma_{zz} \\
\sigma_{yz} \\
\sigma_{zx} \\
\sigma_{xy}
\end{bmatrix} = \dfrac{E}{(1+\nu)(1-2\nu)} \begin{bmatrix}
1-\nu & \nu & \nu & 0 & 0 & 0 \\
\nu & 1-\nu & \nu & 0 & 0 & 0 \\
\nu & \nu & 1-\nu & 0 & 0 & 0 \\
0 & 0 & 0 & \dfrac{1}{2}-\nu & 0 & 0 \\
0 & 0 & 0 & 0 & \dfrac{1}{2}-\nu & 0 \\
0 & 0 & 0 & 0 & 0 & \dfrac{1}{2}-\nu \\
\end{bmatrix} \begin{bmatrix}
\epsilon_{xx} \\
\epsilon_{yy} \\
\epsilon_{zz} \\
\epsilon_{yz} \\
\epsilon_{xz} \\
\epsilon_{xy} \\
\end{bmatrix}
\end{equation}

So the stress and strain tensors can be represented by $ 3 \times 3 $ matrices:
\begin{equation}
\mathbf{\epsilon} = \begin{bmatrix}
\epsilon_{11} & \epsilon_{12} & \epsilon_{13} \\
\epsilon_{21} & \epsilon_{22} & \epsilon_{23} \\
\epsilon_{31} & \epsilon_{32} & \epsilon_{33}
\end{bmatrix}
\end{equation}

\begin{equation}
\mathbf{\sigma} = \begin{bmatrix}
\sigma_{11} & \sigma_{12} & \sigma_{13} \\
\sigma_{21} & \sigma_{22} & \sigma_{23} \\
\sigma_{31} & \sigma_{32} & \sigma_{33}
\end{bmatrix}
\end{equation}

Being a linear mapping between the nine numbers $ \sigma_{ij} $ and the nine numbers $ \epsilon_{kl} $, the stiffness tensor $ c $ is represented by a matrix of $ 3 \times 3 \times 3 \times 3 = 81 $ real number $ c_{ijkl} $. Hooke's law then says that:
\begin{equation}
\sigma_{ij} = - \sum_{k=1}^{3} \sum_{l=1}^{3} c_{ijkl} \epsilon{kl}
\end{equation}


\begin{equation}
\vec{e} = \begin{bmatrix}
e_{xx} \\ 
e_{yy} \\ 
e_{zz} \\ 
e_{xy} \\ 
e_{yx} \\ 
e_{yz} \\ 
e_{zy} \\ 
e_{zx} \\ 
e_{xz}
\end{bmatrix} = \begin{bmatrix}
\partial / \partial x & 0 & 0 \\ 
0 & \partial / \partial y & 0 \\ 
0 & 0 & \partial / \partial z \\ 
\partial / \partial y & 0 & 0 \\ 
0 & \partial / \partial x & 0 \\ 
0 & \partial / \partial z & 0 \\ 
0 & 0 & \partial / \partial y \\ 
0 & 0 & \partial / \partial x \\ 
\partial / \partial z & 0 & 0
\end{bmatrix} \begin{bmatrix}
u_x \\ 
u_y \\ 
u_z
\end{bmatrix} 
\end{equation}

\begin{equation}
\vec{\sigma} = \begin{bmatrix}
\sigma_{11} \\ 
\sigma_{22} \\ 
\sigma_{33} \\ 
\sigma_{12} \\ 
\sigma_{23} \\ 
\sigma_{31}
\end{bmatrix} = \begin{bmatrix}
\sigma_{xx} \\ 
\sigma_{yy} \\ 
\sigma_{zz} \\ 
\sigma_{xy} \\ 
\sigma_{yz} \\ 
\sigma_{zx}
\end{bmatrix} = 
\end{equation}

\section{The Element Stiffness Matrix}
Introducing $ \vec{e} = \mathbf{B} \vec{u} $ and $ \vec{\sigma} = \mathbf{C} \vec{e} $

\end{document}